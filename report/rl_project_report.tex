\documentclass{article}

% if you need to pass options to natbib, use, e.g.:
% \PassOptionsToPackage{numbers, compress}{natbib}
% before loading rl_project.

% to compile a camera-ready version, add the [final] option, e.g.:
 \usepackage[final]{rl_project}

% to avoid loading the natbib package, add option nonatbib:
% \usepackage[nonatbib]{rl_project}

\usepackage[utf8]{inputenc} % allow utf-8 input
\usepackage[T1]{fontenc}    % use 8-bit T1 fonts
\usepackage{hyperref}       % hyperlinks
\usepackage{url}            % simple URL typesetting
\usepackage{booktabs}       % professional-quality tables
\usepackage{amsfonts}       % blackboard math symbols
\usepackage{nicefrac}       % compact symbols for 1/2, etc.
\usepackage{microtype}      % microtypography
\usepackage{graphicx}


% Give your project report an appropriate title!

\title{Comparison of different deep reinforecemnt learning agents playing Starcraft II mini-games using Deepmind's pysc2 API}


% The \author macro works with any number of authors. There are two
% commands used to separate the names and addresses of multiple
% authors: \And and \AND.
%
% Using \And between authors leaves it to LaTeX to determine where to
% break the lines. Using \AND forces a line break at that point. So,
% if LaTeX puts 3 of 4 authors names on the first line, and the last
% on the second line, try using \AND instead of \And before the third
% author name.

\author{
  Oliver Just\\
  \texttt{ojdb20@bath.ac.uk} \\
  \And
  Mahin Ali\\
  \texttt{mra64@bath.ac.uk} \\
  \And
  Arun Kumar\\
  \texttt{ak3304@bath.ac.uk} \\
  \And
  Marcin Tiela\\
  \texttt{mt2303@bath.ac.uk}
  %% examples of more authors
  %% \And
  %% Coauthor \\
  %% Affiliation \\
  %% Address \\
  %% \texttt{email} \\
  %% \AND
  %% Coauthor \\
  %% Affiliation \\
  %% Address \\
  %% \texttt{email} \\
  %% \And
  %% Coauthor \\
  %% Affiliation \\
  %% Address \\
  %% \texttt{email} \\
  %% \And
  %% Coauthor \\
  %% Affiliation \\
  %% Address \\
  %% \texttt{email} \\
}

\begin{document}

\maketitle

\section{Problem Definition}

\section{Background}

\section{Method}

\section{Results}

\section{Discussion}

\section{Future Work}

\section{Personal Experience}


\section*{References}
\small
Vinyals, O., Ewalds, T., Bartunov, S., Georgiev, P., Vezhnevets, A.S., Yeo, M., Makhzani, A., Küttler, H., Agapiou, J.P., Schrittwieser, J., Quan, J., Gaffney, S., Petersen, S., Simonyan, K., Schaul, T., Van Hasselt, H., Silver, D., Lillicrap, T.P., Calderone, K., Keet, P., Brunasso, A., Lawrence, D., Ekermo, A., Repp, J. and Tsing, R., 2017. StarCraft II: a new challenge for reinforcement learning. {\it arXiv (Cornell University) } [Online]. Available from: https://doi.org/10.48550/arxiv.1708.04782.

\normalsize
\newpage
\section*{Appendices}
If you have additional content that you would like to include in the appendices, please do so here.
There is no limit to the length of your appendices, but we are not obliged to read them in their entirety while marking. The main body of your report should contain all essential information, and content in the appendices should be clearly referenced where it's needed elsewhere.
\subsection*{Appendix A: Example Appendix 1}
\subsection*{Appendix B: Example Appendix 2}

\end{document}
